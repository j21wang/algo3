\documentclass{article}
\usepackage{amssymb}
\usepackage{amsmath}
\usepackage{amsfonts}
\usepackage{latexsym}
\usepackage{times}
\usepackage{psfrag,epsfig,epsf}
\usepackage{graphics}
\usepackage{multirow}
\usepackage{fullpage}
\usepackage{verbatim}
\usepackage{fancyheadings}
\usepackage[T1]{fontenc}
\usepackage{arev}
\usepackage{subfigure}
\usepackage{url}
\usepackage[noline,noend,ruled,linesnumbered]{algorithm2e}
\usepackage{algpseudocode}
\linespread{1.02} 

\pagestyle{empty}

\addtolength{\topmargin}{-20pt}
\addtolength{\oddsidemargin}{-5pt}
\addtolength{\textwidth}{20pt}
\addtolength{\textheight}{50pt}

\newenvironment{myitem}{\begin{list}{$\bullet$}
{\setlength{\itemsep}{-0pt}
\setlength{\topsep}{0pt}
\setlength{\labelwidth}{0pt}
%\setlength{\labelsep}{0pt}
\setlength{\leftmargin}{10pt}
\setlength{\parsep}{-0pt}
\setlength{\itemsep}{0pt}
\setlength{\partopsep}{0pt}}}%
{\end{list}}

\begin{document}

\sloppy

\noindent \underline{CS 344: DESIGN AND ANALYSIS OF COMPUTER
  ALGORITHMS \hspace{1.6in} SPRING 2014}

\vspace{0.1in}

\begin{center}
{\bf {\large Homework 3}}
Dynamic Programming and Graph Search
\end{center}

\begin{center}
{\bf Matt Demusz, Frank Porco, Joyce Wang}
\end{center}

\begin{center}
{\bf Part A (35 points)}
\end{center}

\noindent {\bf Problem 1:} \\

\noindent {\bf Assumptions:}
\begin{itemize}
\item Each player only has one possible path to follow.
\item Calculating the distance between two stones happens in constant time.
\item We have a stone.hasNext() that returns true if there is a next stone to jump to
\item We have a stone.next that represents the player jumping to the next stone 
\end{itemize}

\begin{algorithm}[h]
\caption{  Minimum rope length algorithm (Node src1, Node src2)}
\label{algo}
int minLen = distance(src1, src2);\\
int move1, move2, both;\\
Node temp1 = src1;\\
Node temp2 = src2;\\
{\bf {while}} (temp1.hasNext() OR temp2.hasNext()):\\
	\hspace{.5in}{\bf {if}} (temp1.hasNext() AND !temp2.hasNext()):\\
		\hspace{1in}move1 = distance(temp1.next, temp2);\\
		\hspace{1in}{\bf {if}} (move1 > minLen):\\
			\hspace{1.5in}minLen = move1;\\
		\hspace{1in}temp1 = temp1.next;\\
	\hspace{.5in}{\bf {else if}} (!temp1.hasNext() AND temp2.hasNext()):\\
		\hspace{1in}move2 = distance(temp1, temp2.next);\\
		\hspace{1in}{\bf {if}} (move2 > minLen):\\
			\hspace{1.5in}minLen = move2;\\
		\hspace{1in}temp2 = temp2.next;\\
	\hspace{.5in}{\bf {else:}}\\
		\hspace{1in}move1 = distance(temp1.next, temp2);\\
		\hspace{1in}move2 = distance(temp1, temp2.next);\\
		\hspace{1in}both = distance(temp1.next, temp2.next);\\
		\hspace{1in}{\bf {if}} (min(move1, move2, both) == both):\\
			\hspace{1.5in}{\bf {if}} (both > minLen):\\
				\hspace{2in}minLen = both;\\
			\hspace{1.5in}temp1 = temp1.next;\\
			\hspace{1.5in}temp2 = temp2.next;\\
		\hspace{1in}{\bf {else if}} (min(move1, move2, both) == move1):\\
			\hspace{1.5in}{\bf {if}} (move1 > minLen):\\
				\hspace{2in}minLen = move1;\\
			\hspace{1.5in}temp1 = temp1.next;\\
		\hspace{1in}{\bf {else:}}\\
			\hspace{1.5in}{\bf {if}} (move2 > minLen):\\
				\hspace{2in}minLen = move2;\\
			\hspace{1.5in}temp2 = temp2.next;\\
{\bf {return}} minLen;\\
		
\end{algorithm}

\noindent {\bf {Run Time:}}  If we assume the distance function to be constant time, then the overall run time 
of this algorithm would be O(n + m) with n being the number of red stones, and m being the number of blue 
stones.  All of the variable assignments and the checks are constant time, and they run as many times as
the while loop does.  The maximum number of times this loop can run is the case where the shortest 
rope length is when one player jumps, and the other stands still every time.  This in total would be n + m times, leaving the overall running time as O(n + m).

\begin{center}
{\bf Part B (35 points)}
\end{center}

\noindent {\bf Problem 2:}\\

\noindent {\bf A.} You have a collection of $n$ distinct chopsticks of
length $l_{1},\dots,l_{n}$. Any two of them can be paired for use if
the length of them differ at most $k$. How can you easily pair as many
of the chopsticks as possible? Describe a greedy algorithm of time
complexity $O(n\log n)$ to solve this problem and prove the
correctness of your algorithm.\\

\noindent First, we can sort the chopsticks from shortest to longest using mergesort. The runtime of mergesort is $nlogn$. After we sort them, we can start from the shortest chopsticks and grab them in pairs. In the pair that was grabbed, we can take the longer chopstick length and subtract that by the shorter chopstick length. If the result of the subtraction is less than or equal to $k$, then we can put those chopsticks aside as a pair. If the result of the subtraction is greater than $k$, then we can discard the smaller chopstick in the pair and grab the next chopstick in the list of sorted chopsticks to pair with the "greater" chopstick if their difference in length is less than or equal to $k$. Traverse through the list of chopsticks by always looking at the smallest two chopsticks, and repeat the subtraction process above until no chopsticks are left to pair. Traversing through the list of chopsticks requires one iteration only, giving us a running time of $n$, so we have $O(nlogn + n)$, which is equal to $O(nlogn)$.\\

\noindent {\bf B.} Consider now a variant of the above problem. You
can still only pair chopsticks that differ at most $k$ in length. But
now a value $w_{i}$ is also associated with each individual
chopstick. You want to maximize the sum of the values of the
chopsticks that have been paired.\\

\noindent For example, suppose you have 7 chopsticks of length
$5,2,3,11,9,12,16$ and corresponding values $1,1,2,5,3,3,10$. You are
allowed to pair chopsticks that differ by at most 3 units in
length. Then one of the optimal solutions here is $\{ (2,3),(9,11) \}$
of optimal value $1+2+3+5=11$.\\

\noindent How can you pair the chopsticks so as to maximize the value?
Describe a dynamic programming algorithm of time complexity $O(n^{2})$
to solve this problem. Can you do better than $O(n^{2})$?\\

\noindent Given chopsticks with their lengths and weights, we can sort the chopsticks by length in an array $sortedChopsticksArr$ and store its corresponding weight value in it ($O(n)$, where $n$ is the number of chopsticks).Then, we can make a 2D array, called $chopstickSumWeights$. We can use nested loops to iterate through the list of sorted chopsticks: in $sortedChopsticksArr$, for each chopstick $i$, we want to iterate through every chopstick $j$, take the absolute value difference of $sortedChopsticksArr[i]-sortedChopsticksArr[j]$ and check if this value does not exceed $k$ ($O(n^2)$). If the value does not exceed $k$, then we can put in the sum of $sortedChopsticksArr[i]+sortedChopsticksArr[j]$ in $chopstickSumWeights[i][j]$. Otherwise, we just set $chopstickSumWeights[i][j]$ equal to null.\\

\noindent To begin, we can make a new hash table $pairs$, and hash all of the chopsticks as our keys by iterating through our sorted chopsticks list (we can set the values to null for now) ($O(n)$). We can also make a variable called $totalSum=0$ to keep track of the total optimal sum. Now that we have all the values in $chopstickSumWeights$, we have to iterate through it to find the optimal value for the sum of the weights ($O(n^2)$). We can loop through the $i$th row and iterate through the all $j$ columns to find the maximum value in that row. To find the maximum variable, we can have a ChopStick object $currChopstick$ that keeps track of the largest value that has been visited in that row and the other chopstick in that pair.\\

\noindent After we get the maximum value in the $i$th row and the pair of chopsticks for that maximum value, we go into our hash table to check if one of the chopsticks has been used already (if it's value in the hash table is not $null$). If the chopstick is not $null$ and is a ChopStick object, then we compare the sum stored in the hash table for that chopstick to our $currChopstick$ value. If $currChopstick$ value is greater than the sum in the hash table, meaning that we found a better value to replace the old pair, then we go into the hash table, set each chopstick in the old pair to null, subtract the old sum from $totalSum$, and set each chopstick in the new pair's ChopStick object $x$ to the pair's sum and the other chopstick in the pair ($O(1)$). Then, we do $totalSum = totalSum + x$ ($O(1)$). We continue going down the $ith$ rows. If we find that all the values $j$ in the row are $null$, then there is no maximum value, so we just move down to the next row. Continue until we are done iterating through the rows, and $totalSum$ should store the optimal sum at the end. To get the actual chopstick pairs, we can create another array $optimalPairsArr$ that stores all the pairs. We can iterate through $sortedChopsticksArr$ ($O(n)$), and at each chopstick, we go to to the hash table to find if it has a pair ($O(1)$). If it does, then make sure that it already isn't in $optimalPairsArr$ and add that pair into the array.\\

\noindent As noted above, we have $n+n^2+n+n^2+n$. Therefore, we have $O(n^2)$ as the running time, and we cannot do better than $O(n^2)$ since we must use a 2D array to keep track of the weights and we must iterate through this array, which will be $O(n^2)$.\\

\begin{center}
{\bf Part C (20 points)}
\end{center}


\noindent {\bf Problem 3:} In the robotics lab the new robot has just
arrived. The robot has the ability to construct a topological map of
the environment, such as the graph shown in Figure
\ref{fig:problem3}. The robot is allowed to move only forward along
the directions of the edges on the topological map. Moreover, the
graph is being constructed in such a way that will prevent the robot
to execute loops, i.e., the robot is not able to visit a node that it
has already visited.\\

\noindent {\bf A.} Given a start location for your robot and a target
location, provide an efficient algorithm that will return all the
possible paths from the start to the target.  What is the running time
for your algorithm?\\

\noindent {\bf B.} You want to check if the topological map provides
enough information for your robot to be able to visit all the rooms so
as to clean them. Provide an efficient algorithm that will be able to
check if there is a path for the robot on the graph that can visit all
the rooms (i.e., nodes on the graph).\\

\begin{center}
{\bf Part D (20 points)}
\end{center}

\noindent {\bf Problem 4:} You are preparing a banquet where the
guests are government officials from many different countries. In
order to avoid unnecessary troubles, you are asked to check the list
of international conflicts in the last ten years. Then, you will
assign the guests to two tables, such that in each table, any two
guests are not from countries that had conflicts in the last ten
years.\\

\noindent Provide an efficient algorithm that determines whether it is
possible to make such an assignment. If it is possible to do so, the
algorithm should return the assignment of these two tables. What is
the running time?\\

\noindent Given a list of international conflicts of size $n$ and a list of guests from each country of size $m$:\\
\begin{enumerate}
\item Iterate through the list of guests and create a hash map where the key is the contry and the value is the guest's name. ($O(m)$)
\item Create a queue that will be used to store the first node of each disconnected graph.
\item Traverse the list of international conflicts. ($O(n)$) For each country1 and country2 that has a conflict:
\begin{itemize}
\item Use the hash map to check if country1 has a person ($O(1)$). If country1 has a person and is not already on the graph, then add that country's person as a node to the graph ($O(1)$). If country1 has person and is already on the graph or if country1 does not have person, then don't add anything to the graph.
\item Use the hash map to check if country2 has a person ($O(1)$). If country2 has a person and is not already on the graph, then add that country's person as a node to the graph ($O(1)$). If country2 has person and is already on the graph or if country2 does not have person, then don't add anything to the graph.
\item If both nodes are on the graph, then a strongly connected relationship should be denoted between these two nodes.
\item If a disconnected graph has been created, then add one of the nodes from that graph to the queue so that we can keep track of all of the disconnected graphs ($O(1)$).
\end{itemize}
\item Create two arrays: one for table1 and another for table2.
\item Dequeue from the queue ($O(1)$), and starting from that node, and using BFS, traverse through one of the graphs. At the first node of that graph, add it to table1. For all the neighbors of that node, add them to table2 if they do not already exist in table2 since being neighbors denote a conflict, which means that two nodes that are neighbors cannot be in the same table, or array.
\item Continue to traverse using BFS until all nodes have been visited, and everytime we're on a new node, make sure that it does not already exist in a table and that it is not added to the same table as its neighbors that have already been visited. If we find that we cannot add a node to either table because it already has visited neighbors in both tables, then we stop and return false. Otherwise, we continue to dequeue and repeat steps 5 and 6 until the queue is empty, and we return the two arrays of tables.
\end{enumerate}

\noindent Without taking into consideration the running time of BFS, we have $O(m+n)$ so far. For BFS, let $E$ be the count of all edges in the graph, and there are $m$ nodes, or vertices. The total running time is of BFS is $O(|E|+|m|)$. Therefore, we have $O(m+n+E+m) = O(m+n+E)$.




\end{document}

